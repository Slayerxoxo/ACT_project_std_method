%%%%%%%%%%%%%%%%%%%%%%%%%%%%%%%%%%%%%%%%%%%%%%%%%%%%%%%%%%%%%%%%%%%%
%%%%%                       CONCLUSION                         %%%%%
%%%%%%%%%%%%%%%%%%%%%%%%%%%%%%%%%%%%%%%%%%%%%%%%%%%%%%%%%%%%%%%%%%%%
%%%%%                Author : Coraline Marie                   %%%%%
%%%%%%%%%%%%%%%%%%%%%%%%%%%%%%%%%%%%%%%%%%%%%%%%%%%%%%%%%%%%%%%%%%%%



Les résultats précédents démontrent que la méthode standard de Fung et de McKeown reste très limité. Sur les corpus comparables du cancer du sein, seulement 28 \% des meilleures suggestions sont les bonnes traductions. Cependant, l'ajout des différents dictionnaires de cognats démontrent que ces résultats peuvent être légèrement amélioré, jusqu'à 33 \%.\\
	
	Néanmoins, il existe plusieurs pistes pour améliorer ces résultats. Tout d'abord, les corpus comparables utilisés sont de petites taille. L'utilisation de corpus plus importants augmenterais la précision des vecteurs de contextes, ce qui améliorerait également la précision de l'évaluation. De plus, il a été constaté que certains termes techniques tel que \textit{letrozole} ou \textit{raloxifene} sont des transfuges. Il est donc logique de penser que l'ajout d'un dictionnaire de transfuge en complément du dictionnaire classique augmenterait la précision des résultats.
