%%%%%%%%%%%%%%%%%%%%%%%%%%%%%%%%%%%%%%%%%%%%%%%%%%%%%%%%%%%%%%%%%%%%
%%%%%                     INTRODUCTION                         %%%%%
%%%%%%%%%%%%%%%%%%%%%%%%%%%%%%%%%%%%%%%%%%%%%%%%%%%%%%%%%%%%%%%%%%%%
%%%%%                Author : Coraline Marie                   %%%%%
%%%%%%%%%%%%%%%%%%%%%%%%%%%%%%%%%%%%%%%%%%%%%%%%%%%%%%%%%%%%%%%%%%%%

Principalement utilisée pour la traduction automatique ou pour la recherche d'informations, \textit{l'alignement de chaînes et de textes} est une discipline de traitement des langues, qui permet de mettre en correspondance des unités textuelles, par processus automatiques.\\

La traduction automatique est un outil aujourd'hui commun, pratique, et facilement accessible sur Internet. Cependant, il reste encore imparfait, car il n'est pas capable de donner de bonnes traductions dans toutes les langues, et pour tous les mots. Les termes techniques sont notamment difficiles à traduire, car ils sont rarement utilisés, et également peu présents dans les dictionnaires.\\

La méthode standard de Pascale Fung et de Kathleen McKeown\cite{Fung97findingterminology} propose un algorithme qui permet de traduire des termes techniques inconnus, à l'aide de corpus comparables. Ce rapport présente donc les résultats de l'implémentation de cette méthode sur deux corpus comparables (français et anglais), ainsi que le travail de prétraitement des corpus, réalisé dans le but d'améliorer les résultats de traduction de la méthode standard.

